% !TEX root = aws2014.tex

\section{Selmer group heuristics and sieves}
\thanksauthor{Bjorn Poonen}





\subsection{Sieves in arithmetic geometry}

Sieves form a very large topic, so we will only touch on a specific example, 
called the \emph{closed point sieve}. We'll start by counting square-free 
integers. For a subset $\cP\subset \dN$, the \emph{density} of $\cP$, written 
$\prob(\cP)$, is the limit 
\[
  \prob(\cP) = \lim_{B\to \infty} \frac{\# \cP\cap [1,B]}{B} .
\]

\begin{theorem}
$\prob(n\text{ is squarefree})=\prod_p (1-p^{-2}) = \zeta(2)^{-2} = \frac{6}{\pi}$. 
\end{theorem}

\begin{proof}
The basic idea is to ``sieve out the integers divisible by $p^2$,'' one prime 
a time. Choose cutoffs $r$ and $\sqrt B$. We consider primes $\leq p$ to be 
``small,'' primes between $r$ and $\sqrt B$ ``medium,'' and primes $>\sqrt B$ 
to be ``large.'' Fix $r$. Then 
\[
  \prob(p^2\nmid n\text{ for all $p\leq r$}) = \prod_{p\leq r} \left(1-\frac{1}{p^2}\right) .
\]
All that remains is to show that 
\[
  \lim_{r\to \infty} \bar\prob(n\text{ is divisible by $p^2$ for some $p>r$}) = 0 .
\]
Concretely, this is saying that 
\[
  \lim_{r\to\infty} \limsup_{B\to \infty} \frac{\#\{n\leq B:n\text{ is divisible by $p^2$ for some $p>r$}\}}{B} = 0 .
\]
Since all we need is an upper bound, we can use stupid upper bounds. We can 
start out with the medium primes. We have 
\begin{align*}
  \# \{n\leq B:n\text{ is divisible by $p^2$ for some $p\in (r,\sqrt B]$}\} 
    &\leq \sum_{p\in (r,\sqrt B]} \left\lfloor\frac{n}{p^2}\right\rfloor \\
    &\leq \sum_{p\in (r,\sqrt B)} \frac{B}{p^2} \\
    &\leq B\int_r^\infty \frac{dx}{x^2} \\
    &= \frac{B}{r} .
\end{align*}
All that remains is the large primes. But 
\[
  \#\{n\leq B:n\text{ is divisible b $p^2$ for some $p>\sqrt B$}\} = 0 ,
\]
so we are done. 
\end{proof}

Next we look at square-free values of a polynomial, e.g. 
$\prob(n^4+1\text{ is square-free})$. 

\begin{conjecture}
The probability is $\prod_p \left(1-\frac{c_p}{p^4}\right)$, where 
$c_p=\#\{n\in \dZ/p^2:n^4+1\equiv 0\}$. 
\end{conjecture}
\begin{proof}[Proof?]
We start with small primes as before. We get 
\[
  \prob(p\nmid n^4+1\text{ for all $p\leq r$}) = \prod_{p\leq r}\left(1-\frac{c_p}{p^2}\right) .
\]
Let's attack the large primes. Here \emph{large} means $p>B^2$. As before, 
\[
  \#\{n\leq B:p^2\nmid n^4+1\text{ for some $p>B^2$}\} =0 .
\]
All that remains are the medium primes, where \emph{medium} means 
$r<p\leq B^2$. If we try to repeat the na\"ive bound, we get, for each $p$ 
not dividing $\disc(x^4+1)$, 
\[
  \#\{n\leq B:p^2\mid n^4\} \leq 4\cdot \left\lceil\frac{B}{p^2}\right\rceil
\]
It follows that 
\begin{align*}
  \#\{n\leq B:p^2\mid n\text{ for some $p\in (r,B^2)$}\} 
    &\leq \sum_{p\in (r,B^2)} 4\left\lceil\frac{B}{p^2}\right\rceil \\
    &\leq \sum_{p\in (r,B^2)} \left(4\frac{B}{p^2} + 4\right) \\
    &\leq \frac 4 r B + 4\frac{B^2}{\log B^2} ,
\end{align*}
which is \emph{not} small enough relative to $B$ for the proof to work. 
\end{proof}

This conjecture follows from the ABC conjecture due to the work of many people 
(Browking, Filaseta, Greaves, Schinzel, Granville). They worked with arbitrary 
cyclotomic polynomials. 





\subsection{Closed points and zeta functions}

Let $k$ be a field, and $X$ a scheme of finite type over $k$. A \emph{closed 
point} of $X$ is exactly that -- a point in $X$ which is closed. In an affine 
chart $\spec(A)$, closed points $P$ correspond to maximal ideals 
$\fm\subset A$. Let $\kappa(P)$ be the \emph{residue field} of $P$, i.e. the 
field $A/\fm$. The weak nullstellensatz tells us that $\kappa(P)$ is a finite 
extension of $k$. These notions also make sense if $X$ is a scheme of finite 
type over $\spec(\dZ)$. In that case, the residue fields will be finite. 

If $X/k$ is of finite type and $P$ is a closed point in $X$, then we define the 
\emph{degree} of $P$, written $\deg P$, to be the integer $[\kappa(P):k]$. 
If $\kappa(P)=A/\fm$ for some affine chart $\spec(A)$, then 
$\deg P=[A/\fm:k]$. 

\begin{example}
A closed point on $\dA^1$ corresponds to a maximal ideal of $k[t]$. Such ideals 
are generated by unique monic irreducible polynomials in $k[t]$, which are 
in bijection with $\gal(\bar k/k)$-orbits in $\dA^1(\bar k)$. 
\end{example}

More generally, if $X$ is a $k$-scheme, a closed point of $X$ will correspond 
to a $\gal(\bar k/k)$-orbit in $X(\bar k)$. Write $|X|$ for the set of closed 
points in $X$. 

We can reinterpret the Riemann zeta function as 
\[
  \zeta_{\spec\dZ}(s) = \prod_{p\text{ prime}} (1-p^{-s})^{-1} = \prod_{P\in |\spec \dZ|} (1-\#\kappa(P)^{-s})^{-1} .
\]
In general, if $X$ is a scheme of finite type over $\dZ$, we define 
\[
  \zeta_X(s) = \prod_{P\in |X|} (1-\# \kappa(P)^{-s})^{-1} .
\]
This Euler product converges for $\Re(s)>\dim X$. As a special case, any scheme 
of finite type over $\dF_q$ has a zeta function. In that case, 
$\#\kappa(P)=q^{\deg P}$, so 
\[
  \zeta_X(s) = \prod_{P\in |X|} (1-q^{-s\deg P})^{-1} .
\]
As a power series in $T=q^{-s}$, this turns out to be 
\[
  Z_X(T) = \exp\left(\sum_{r\geq 1} \frac{\# X(\dF_{q^r})}{r} T^r \right) .
\]





\subsection{Bertini smoothness theorems}

Let $X\subset \dP^n$ be defined over a field $k$. Suppose $X$ is smooth of 
dimension $m$ over $k$. We only assume $X$ is quasi-projective. Then there 
exists a dense open $U\subset (\dP^n)^\vee$ such that for all $u\in U$, 
the corresponding hyperplane $H_u\subset \dP_{\kappa(u)}^n$ has $H_u\cap X$ 
smooth of dimension $n-1$ over $\kappa(u)$. 

\begin{corollary}
If $k$ is infinite, there exists $H/k$ such that $H\cap X$ is smooth. 
\end{corollary}

If $k$ is finite, this corollary is wrong! To fix this, we will change the 
problem. Let $S=\dF_q[x_0,\dots,x_n]$ and $\dP^n=\proj(S)$. Let 
\[
  S_d = \{\text{homogeneous polynomials of degree $d$ in $S$}\} ,
\]
and put $S_{\hom}=\bigcup_{d\geq 0} S_d$. For $f\in S_d$, let 
$H_f = \proj(S/f)$. For $\cP\subset S_{\hom}$, define the \emph{density} of 
$\cP$ to be 
\[
  \mu(\cP) = \lim_{d\to \infty} \frac{\#\cP\cap S_d}{\# S_d} .
\]

\begin{theorem}[Bertini smoothness over $\dF_q$]
Let $X$ be a smooth $m$-dimensional quasi-projective subscheme of 
$\dP^m$ over $\dF_q$. Then 
\[
  \cP=\{f\in S_{\hom}:H_f\cap X\text{ is smooth of dimension $m-1$}\} 
\]
has density $\mu(\cP) = \zeta_X(m+1)^{-1}\in \dQ\cap [0,1]$. 
\end{theorem}

For comparison, 
\[
  \prob(n\ne 0\text{ in }\spec\dZ\text{ is regular}) = \zeta(2)^{-1} 
\]
where $2=\dim(\spec\dZ)+1$. There is a conjectural joint generalization of 
these two facts. 

\begin{example}
Let $X+\dP^2$ over $\dF_2$. The question in this example is: what is the 
probability that a plane curve is smooth? It turns out that 
$\#\dP^2(\dF_{2^r}) = 4^r+2^r+1$, so 
$Z_X(T) = \left((1-T)(1-2 T)(1-4 T)\right)^{-1}$, so we get the following 
theorem. 
\end{example}

\begin{theorem}
$\mu(\text{smooth plane curves in $\dP^2$ over $\dF_2$}) = \frac{21}{64}$. 
\end{theorem}




