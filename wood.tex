% !TEX root = aws2014.tex

\section{Asymptotics for number fields and class groups}
\thanksauthor{Melanie Matchett Wood}





\subsection{Counting number fields}

\begin{theorem}[Hermite]
Given $X>0$, there are finitely many number fields $K$ (up to isomorphism, or 
in $\bar\dQ$) with $|\disc K|<X$. 
\end{theorem}

\begin{question}
What are the asymptotics in $X$ of the function 
$N(X)=\#\{K:|\disc K|<X\}$?
\end{question}

These types of questions turn out to be easiest to address after restricting to 
number fields with some fixed invariant. The most obvious of these is the 
\emph{Galois group}. If $K$ is a number field of degree $n$, the Galois group 
of $K$, denoted $\gal(K)$, is the image of $\gal(\tilde K/\dQ) \to S_n$, where 
$\tilde K$ is the Galois closure of $K$. The representation 
$\gal(\tilde K/\dQ) \to S_n$ is given by the action of $\gal(\tilde K/\dQ)$ on 
the $n$ homomorphisms $K\to \bar\dQ$. 

\begin{example}
Write $K=\dQ(\theta)$, and let $\theta_1,\dots,\theta_n$ be the $n$ conjugates 
of $\theta$ in $\bar\dQ$. Then we are interested in the action of 
$\gal(\tilde K/\dQ)$ on the set $\{\theta_1,\dots,\theta_n\}$. 
\end{example}

So for us$, \gal(K)$ is not just an abstract group -- it is a group of 
permutations.  

\begin{example}
Let $K$ be a cubic field. Then $\gal(K)\subset S_3$. We call $K$ a \emph{cyclic 
cubic field} if $K$ is Galois and $\gal(K)=A_3\simeq \dZ/3$. If $K$ is 
non-Galois, then $\gal(K)=S_3$. 
\end{example}

We are interested in the asymptotics of the function 
\[
  N_\Gamma(X) = \#\{K:|\disc K|<X\text{ and }\gal(K)\simeq \Gamma\} .
\]





\subsection{Local behavior}

Given a place $p$ of $\dQ$ (so $p$ is a prime or $p=\infty$), we can form the 
completion $K_p = K\otimes_\dQ \dQ_p$. (Recall that $\dQ_\infty=\dR$). This is 
bigger than the ``honest completion'' of $K$ at a place of $K$. If $p$ is a 
finite prime, we would have $(p)=\fp_1^{e_1}\cdots \fp_r^{e_r}$ in $\fo_K$. The 
ring $K_p$ is a direct product of field extensions of $\dQ_p$. More precisely, 
\[
  K_p = \prod_i K_{\fp_i} ,
\]
where $K_{\fp_i}$ is the completion of $K$ at the prime $\fp_i$. So $K_p$ is an 
\'etale $\dQ_p$-algebra. The ring $K_p$ captures the splitting, inertia, etc.\ 
or $p$ in $K$. 

\begin{question}
What are the asymptotics of 
\[
  N_{\Gamma,M}(X) = \#\{K:|\disc K|<X,\gal(K)\simeq \Gamma\text{, and }K_p\simeq M\} ,
\]
for some group $\Gamma$ and \'etale $\dQ_p$-algebra $M$?
\end{question}

\begin{example}
How many quadratic number fields are there split completely at $7$? This is 
essentially the study of the function $N_{S_2, \dQ_7^{\times 2}}$. 
\end{example}

We can also ask probabilistic questions. Define 
\begin{align*}
  \dP_{\disc} &(\text{quadratic $K$ split completely at $7$}) \\
    &= \lim_{X\to \infty} \frac{\#\{K:|\disc K|<X, \gal(K)\simeq S_2\text{ and }K\text{ s.c.\ at $7$}\}}{\#\{K:|\disc K|<X\text{ and }\gal(K)\simeq S_2\}}
\end{align*}




\subsection{Independence}

Are the probabilities above independent of the prime involved? Consider the 
following diagram:
\begin{center}
\begin{tabular}{c|cccc}
  & 2 & 3 & 5 & 7  \\ \hline 
  $\dQ(\sqrt{-3})$ & i & r & i & s \\
  $\dQ(i)$ & r & i & s & i \\
  $\dQ(\sqrt 5)$ & i & i & r & i \\
  $\dQ(\sqrt{-7})$ & s & i & i & r
\end{tabular}
\end{center}

Here ``s'' denotes ``split,'' ``i'' denotes ``inert,'' and ``r'' denotes ``ramified.'' 
\v Cebotarev's density theorem tells us that if we look in a single row, 
we get $\frac 1 2$ split, none (probabilistically) ramified, and one-half 
inert. On the other hand, 
\begin{align*}
  \dP_{\disc}(\text{$K$ quadratic split at $7$}) 
    &= \frac{7}{16} \\
  \dP_{\disc}(\text{$K$ quadratic inert at $7$}) 
    &= \frac{7}{16} \\
  \dP_{\disc}(\text{$K$ quadratic ramified at $7$}) 
    &= \frac{1}{8} .
\end{align*}

So \emph{\v Cebotarev independence} for us means the (asymptotic) independence 
of the rows in the chart. This is known. More generally, if we listed all 
(Galois) number fields in the rows, we have \emph{\v Cebotarev dependence} 
between two fields if and only if they have no common subfield larger than 
$\dQ$. 

\begin{question}
What do we expect for primes?
\end{question}

In other words, if we fix a prime $p$, how does the ramification data above $p$ 
vary with the fields. 





\subsection{Counting class groups}

Let's begin with a precise question. Consider imaginary quadratic fields. 

\begin{question}
Given an odd prime $p$ and a finite abelian $p$-group $G$, what proportion of 
imaginary quadratic fields $K$ (ordered by discriminant) have Sylow 
$p$-subgroup of the $\class(K)$ isomorphic to $G$?
\end{question}

Recall that the \emph{class group} of $K$, $\class(K)$ is a finite abelian 
group. Such groups are the direct sum of their Sylow $p$-subgroups. So we are 
interested in what, asymptotically, the $p$-part of $\class(K)$ looks like. We 
restrict to odd primes because \emph{genus theory} tells us something about the 
case $p=2$. We can also ask for averages of other $f$ over class groups (where 
$f$ is a characteristic function). 

\begin{example}
What are 
\begin{align*}
  \lim_{X\to \infty} &\frac{\sum_K \left(\frac{\# \class(K)}{p\cdot \class(K)}\right)^k}{\#\{K:\text{$K$ imaginary quadratic and $|\disc K|<X$}\}} \\ 
  \lim_{X\to \infty} &\frac{\sum K \#\operatorname{Sur}(\class(K), A)}{\text{same}} ?
\end{align*}
were $A$ is a finite abelian group and $\operatorname{Sur}(A,B)$ is the number 
of surjections $A\to B$. 
\end{example}

For a function $f$ on finite abelian groups, write $M_\text{field}(f)$ for this 
average. 





\subsection{Cohen-Lenstra Heuristics}

The main idea is that ``some things in nature'' occur with frequency inversely 
proportional to their number of automorphisms. 

\begin{example}
Consider cubic fields in $\bar\dQ$. The Galois fields occur appear once, but 
have three automorphisms, while the non-Galois fields appear three times, but 
have only one automorphism. 
\end{example}

\begin{conjecture}[Cohen-Lenstra, Gerth for $p=2$]
For any ``reasonable'' $f$, we have 
\[
  M_\text{field}(f) = \lim_{n\to \infty}\frac{\displaystyle\sum_\text{fin.\ ab. gp.\ $|\cdot |\leq n$} \frac{f(G)}{\#\aut(G)}}{\displaystyle\sum_\text{fin.\ ab. gp.\ $|\cdot |\leq n$} \frac{1}{\#\aut(G)}}
\]
where the average is taken over $2\class(K)$. 
\end{conjecture}

Cohen and Lenstra compute $M_\text{group}(f)$ for many examples of $f$. 

\begin{example}
If $f$ is the characteristic function of the ``odd cyclic part,'' then 
$M_\text{group}(f)\approx 0.977575\ldots$. 
\end{example}

\begin{example}
If $A$ is a finite abelian group and $f(G)=\#\operatorname{Surj}(G,A)$, then 
$M_\text{group}(f) = 1$. 
\end{example}

\textbf{\ldots stopped taking notes\ldots}




